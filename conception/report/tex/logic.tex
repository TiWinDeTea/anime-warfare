\chapter{Logique} \label{chapter:logique}
\section{Introduction}
Lorsque nous avons conçu notre logique, nous avons choisi de la diviser en 2
parties majeures, l'une chargée de décrire l'état actuel du jeu, la deuxième
étant chargée de contrôler la première.

Pour ce faire, nous avons décidé d'utiliser une state machine pour ce jeu.
Nous avons donc divisé le jeu en plusieurs états, qui sont les phases du jeu original.
La state machine logique fonctionne de la même manière que celle des menus (voir chapitre \ref{chapter:menu}).

\section{Diagramme états-transitions}

\mfigure[H]{width=\linewidth}{sm_logic}{Diagramme états-transitions de la logique}{sm::logic}

Le diagramme ci-dessus présente le fonctionnement global de la logique à travers
les changements de phases du jeu.

Lorsque le jeu est lancé, nous rentrons automatiquement dans la phase « Recrutement du staff »
qui effectue le calcul du nombre d'employés des joueurs.
Une fois ce calcul fait, on passe à l'état de la phase « Détermination du premier joueur ».

La détermination du premier joueur se fait selon les règles, une fois ce joueur
déterminé, on passe à la phase suivante, la phase d'action si l'on est au premier
tour, sinon la phase de marketing.

La phase d'action est-elle même composée d'une state-machine pour simplifier la
réalisation de certaines actions, comme les batailles.

Pour la phase d'action, comme pour la phase de marketing, on passe à l'état
suivant lorsque tous les joueurs ont joué, ou si l'on a terminé la partie.
Dans le cas où l'on a terminé la partie, on passe dans l'état final dans lequel
on détermine les éventuels gagnants.

\section{Diagrammes de classes}
\subsection{États logiques}

Le diagramme de classe suivant correspond à l'implémentation de la state machine
dans la logique.

\mfigure[H]{width=\linewidth}{cl_logic_states}{Diagramme de classe des états}{cl::logic_states}

L'état final est ici représenté par l'état \textit{GameEndedState}, qui s'occupe de
déterminer les vainqueurs éventuels de la partie.
De plus nous avons choisi de différencier l'implementation de la phase de
recrutement en fonction du tour pour éviter de vérifier à quel tour on est à chaque fois.
Il en est de même pour la phase de sélection du premier joueur. \newline
De part la spécificité du premier tour, certains états ont été spécialisé.

Nous avons choisi de créer un état abstrait \textit{GameState} pour
représenter un état du jeu générique qui contient une référence sur le
\textit{GameBoard}\footnote{Le GameBoard est une classe représentant le plateau de jeu.}
qui permet aux états d'intéragir sur le jeu.

\subsection{Données de la logique}

Dans l'UML suivant, on retrouve les classes qui ne contiennent que des données,
ainsi que d'autres systèmes nécessaires pour gérer le jeu (par traitement externe).
Ces classes sont les suivantes:
\begin{itemize}
    \item \textit{Studio}
    \item \textit{FactionType}
    \item \textit{Player}
    \item \textit{UnitBuffedCharacteristics}
    \item \textit{UnitBasicCharacteristics}
    \item \textit{UnitCostModifier}
    \item \textit{UnitCounter}
    \item \textit{Unit}
    \item \textit{UnitType}
    \item \textit{Zone}
    \item \textit{UnitType}
    \item \textit{GameMap}
    \item \textit{UnitLevel}
    \item \textit{GameBoard}
    \item \textit{MarketingLadder}
\end{itemize}

\mfigure[H]{width=\linewidth}{cl_logic}{Diagramme de classe de la logique}{cl::logic}

On retrouve également deux autres systèmes permettant de gérer le jeu,
le premier étant le système des buffs. Chaque joueur, représenté par la classe
\textit{Player}, possède un \textit{BuffsManager} qui permet de gérer les buffs du joueur.
Les buffs sont ajoutés dans les \textit{BuffsManager} des joueurs concernés par
celui-ci.

\textit{Buff} est une classe abstraite dont l'implémentation se doit d'écouter des évenements,
afin de s'appliquer lorsque les conditions nécessaires sont remplies.
La classe \textit{BuffMask} permet de modifier les caractéristiques d'une unité en
l'ajoutant dans la classe \textit{UnitBuffedCharacteristics} d'un joueur.
La classe \textit{UnitBasicCharacteristics} définit les caractéristiques de base de l'unité.

Les classes \textit{Buff}, \textit{BuffMask}, \textit{UnitBuffedCharacteristics} sont dans le même package (qui est
différent de celui de toutes les autres classes). Ainsi, certaines méthodes et attributs sont en visibilité \emph{package}
et ne peuvent être modifiés que par les classes appropriées.

Les \textit{SelfActivables} et \textit{Activables} permettent de débloquer les capacités (\textit{Capacity}) des
joueurs au cours de la partie, lorsque les conditions nécessaires à l'activation
de la capacité sont remplies.

