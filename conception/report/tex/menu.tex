\chapter{Menus d'accueil}\label{chapter:menu}

La figure \figref{menu-statesmachine} est le diagramme états-transitions des menus d'accueil.
Ce diagramme décrit les différents états possibles :
\begin{itemize}
    \item \emph{Menu principal} ; permet de transiter vers les autres menus spécialisés ou de fermer le programme.
    \item \emph{Options} ; permet le réglage des paramètres du client de jeu.
    \item \emph{Liste des serveurs} ; permet de choisir une partie à rejoindre ou d'en créer une.
    \item \emph{Création du serveur} ; permet de configurer le serveur à créer.
    \item \emph{Salon de jeu} ; permet de discuter avec les autres joueurs et de lancer le jeu.
    \item \emph{Jeu} ; une autre state machine prend le relais et gère le jeu en lui-même. %TODO : comment passer de 'jeu' à l'état terminal ?
\end{itemize}

\mfigure[H]{width=0.95\linewidth}{sm_menu.png}{Diagramme états-transitions des menus d'accueil}{menu-statesmachine}

\newpage
La figure \figref{menu-classes} est le diagramme des classes des menus d'accueil.
Pour gérer les différents états présentés plus haut, nous avons choisi d'utiliser une machine à état
(state machine\footnote{automate fini}).
La classe \emph{DefaultStateMachine} est l'implémentation par défaut de l'interface \emph{StateMachine}.
Elle est suffisante dans la plupart des cas.
Tous les états de la machine à état sont des \emph{MenuState} qui ont comme particularité de tous avoir
le \emph{rootLayout} dans lequel charger le contenu du menu.
La classe \emph{MainApp} (qui est l'application javafx) créé l'état initial et initialise la state machine avec
le-dit état au lancement.
Les états se créeent ensuite les uns les autres : lorsque l'on clic sur le bouton option, le \emph{MainMenuState} créé un
\emph{SettingsMenuState} qui sera renvoyé par la méthode \emph{next}. Une fois que la state machine sort de l'état
\emph{MainMenuState}, l'état est \emph{consommé} (détruit). C'est aussi de cette façon que la state machine de
la logique fonctionne (voir chapitre \ref{chapter:logique}).

\mfigure[H]{width=0.95\linewidth}{cl_menu.png}{Diagramme de classes des menus d'accueil}{menu-classes}

