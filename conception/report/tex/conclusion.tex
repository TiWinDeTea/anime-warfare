\chapter{Conclusion}

Nous avons réussi à implémenter toutes les fonctionnalités majeures du projet.
Nous sommes satisfaits de notre conception au niveau de la logique, mais s'il fallait refaire certaines parties du
projet, ça serait au niveau du GUI (tout ce qui est in game). Le GUI a été implémenté tardivement dans le projet
et n'a pas bénéficié d'autant de réflexion que le reste du projet (très peu d'UML à ce sujet, …).
D'autre part, au niveau du réseau, la procédure de mise en place des événements est très lourde :
il faut souvent créer un ou deux événements (et les listeners associés) : un côté serveur et un côté client.
Puis, il faut créer une classe intermédiaire qui doit contenir les informations nécessaires vis à vis de l'événement
(le tout serializable). Cette classe doit être enregistrée au près du serveur et du client.
Puis il faut implémenter les listeners côté client ou côté serveur. On peut enfin utiliser l'événement pour
implémenter des fonctionnalités concrètes. Nous trouvons cette procédure assez contraignante. C'est aussi
dû à la librairie tierce utilisée (Kryonet).
Enfin, même si nous avons eu le temps d'implémenter une bonne partie des buffs et des capacités spéciales,
elles ne sont pas encore toutes implémentées. Mais le système est suffisament extensible grâce à un jeu
conséquent d'événements internes.

