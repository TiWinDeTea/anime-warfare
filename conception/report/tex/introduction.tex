\chapter{Introduction}

    \section{Principe de base}

        Pour notre projet, nous avons décidé d'adapter le jeu de plateau Cthulhu Wars
        sous forme d'un jeu informatique, Anime Warfare. Le but de ce jeu est simple :
        être le premier à obtenir le maximum de fans tout en réalisant ses six productions
        majeures. Chaque joueur joue chacun son tour durant diverses phases, et peut obtenir
        des capacités uniques, liés à sa license (son ''équipe'').
        \newline
        Chaque joueur dispose d'unités qu'il peut, moyennant un coût, placer sur une
        carte ; les unités ainsi placée s'affrontent alors dans des batailles d'insultes
        sanglantes, dans le but d'humilier leurs adversaires à mort.


    \section{Détails des règles}

        \subsection{Tour de jeu}

            Un tour de jeu est décomposé en 4 phases :

            \begin{itemize}
                \item La phase de recrutement du staff.
                \item La phase de détermination du premier joueur
                \item La phase de marketing
                \item La phase d'action
            \end{itemize}

            Après avoir effectué la dernière phase du jeu, on retourne à la première phase.
            À noter toute fois que la phase de marketing n'est pas joué durant le premier tour de jeu.

        \subsection{Recrutement du staff}
            Durant cette phase, on détermine le nombre d'employés de chaque joueur. Ces employés seront utilisé
            pendant les autres phases pour effectuer des actions.

            \subsubsection{Détail du calcul}
                Le nombre d'employés qu'un joueur recrute dépend du nombre de studios et de mascottes sur la carte.
                \begin{itemize}
                    \item Une mascotte alliée raporte 1 employé.
                    \item Une mascotte ennemie capturée raporte 1 employé.
                    \item Un studio indépendant rapport 1 employé.
                    \item Un studio contrôlé rapporte 2 employés.
                \end{itemize}

                Si un joueur recrute moins de la moitié des employés du joueur ayant le plus d'employés, on augmente
                le nombre de membres de son staff pour qu'il en ai au moins la moitié (arrondi au supérieur).

        \subsection{Détermination du premier joueur}

            Le premier joueur pour un tour donné est déterminé de la façon suivante :
            \begin{itemize}
                \item Au premier tour, le premier joueur est choisi aléatoirement
                \item Aux autres tour, le joueur ayant le plus d'employés commence
                \item En cas d'égalité, le premier joueur du tour précédent départage.
            \end{itemize}

            Le premier joueur est aussi celui qui choisira le sens de rotation du tour de jeu.

        \subsection{Marketing}

            Durant la phase marketing, en mode partie courte, chaque joueur gagne 1k fans par
            studio contrôlé.
            \newline
            Chacun leur tour,les joueurs peuvent également employer du staff pour effectuer
            une convention, et ainsi gagner en popularité (ie : gagner des fans).
            Un joueur ne peut effectuer \emph{une seule} convention.
            \newline
            En débutant par le premier joueur, chacun choisi s'il souhaite effectuer une convention.
            \subsubsection{Organiser une convention}

                Le coût d'une convention est déterminé en regardant la \textit{piste Marketing}. Il s'agit
                d'une échelle sur laquelle sont placés des nombres indiquant le coût des conventions, et d'un
                curseur, qui est placé au début du jeu au début de la piste.
                \\
                Comment l'effectuer :
                    \begin{itemize}
                        \item On utilise autant d'employés qu'indiqués par le curseur de la piste marketing.
                        \item On avance le curseur d'un cran.
                        \item Le joueur gagne alors 1k fans pour chaque studio qu'il contrôle et gagne un droit de campagne publicitaire pour chacun de ses Héros en jeu.
                    \end{itemize}

                Si le curseur atteint la fin de la piste, l'évènement \textit{Crise économique du secteur} est déclenché.
                Le jeu se termine, et on tente de déterminer un gagnant. Dans certains cas, il peut n'y avoir aucun gagnant.

            \subsubsection{Évènements spéciaux}

                Certaines capacités et productions prennent effet durant la phase marketing.
                Sauf cas exceptionnel, les effets surviennent après les conventions.
                En cas de conflit, c'est dans l'ordre du tour de table que les effets sont appliqués.


        \subsection{Campagne publicitaires}

            Lorsque l'on gagne un droit de campagne publicitaire, on le pioche dans la réserve
            (sans le révéler aux autres joueurs). La valeur du droit de campagne (entre 1000 et 3000 fans)
            n'est connue que du joueur. À tout moment, le dit joueur peut décider de révéler ses
            droits et ainsi gagner leur valeur en fans.

        \subsection{Progression sur l'échelle de popularité}

            Le plateau de jeu dispose d'un échelle de popularité, allant de 0k fans à 30k.
            Chaque joueur dispose de sa propre jauge de fans (représentant sa position sur l'échelle),
            qui augmente de 1 lorsqu'il obtient 1000 fans.
            \\
            Avoir un maximum de fans est une des conditions de victoires.



        \subsection{Phase d'action}

            Pendant la phase d'action, chaque joueur doit effectuer une action, en respectant l'ordre de jeu.
            \\
            Trois types d'actions sont possible :
            \begin{itemize}
                \item Les \textit{actions Communes}
                \item Les \textit{actions Uniques}
                \item Les \textit{actions Illimités}
            \end{itemize}

            Les actions communes sont celles disponibles pour toutes les licenses.
            On ne peut en effectuer qu'un par tour.
            \\
            Les actions uniques, quant à elles, sont unique à chaque license.
            Dans la majorité des cas, on ne peut en faire qu'une part tour
            \\
            Enfin, les actions illimités sont celles effectuables autant de fois que
            l'on veut, avant et/ou après son action commune/unique.
            \newline
            La phase d'action continue tant qu'il reste au moins un joueur ayant au moins un employé disponible.
            Si un joueur n'a plus aucun employé disponible, il ne peut plus effectuer d'actions (même gratuite)
            jusqu'à la fin de la phase.
            Un joueur peut passer son tour, cela fait immédiatement prendre congé à tout son staff.
            \newline
            Lors d'un round de la phase, le joueur concerné \emph{doit} effectuer une action commune
            ou une action unique mais ne peut pas faire les deux.

            \subsubsection{Liste des actions communes}

                \begin{itemize}
                    \item 3 membres du staff – ouvrir un studio. \\
                        Il est nécessaire de disposer d'une mascotte dans la \emph{zone sans studio} où vous voulez ouvrir votre studio.
                    \item N membre(s) du staff – Déplacer N unités d'une case.
                    \item 1 membre du staff – Engager un affrontement. \\
                        Cette action devient \textit{pseudo illimité} quand la license possède ses 6 productions \\
                        Il n'est \emph{pas possible de lancer plus d'un affrontement par zone} lors de son round.
                    \item 1 membre du staff – Capturer une mascotte. \\
                        Pour capturer un adorateur, il faut une unité supérieure aux adversaires de la faction concernée de la zone. \\
                        L'adorateur ne peut pas se défendre même s'il a des points de combat.
                    \item 1 membre du staff – Recruter une mascotte. \\
                        Ne peuvent apparaître que dans une zone où il y a déjà une unité du joueur ou
                        n'importe où si le joueur n'a plus d'unité.
                    \item ? membre(s) du staff – Dessiner un personnage. \\
                        Un personnage ne peut apparaître que dans une zone ayant un studio contrôlé par le joueur.
                    \item ? membre(s) du staff – Dessiner un Héros. \\
                        Voir les spécificités de chaque héros
                    \item Tous son personnel – passer son tour.
                \end{itemize}


            \subsubsection{Affrontement publicitaire}
                Les affrontements sont des concours d'insulte et se déroulent en trois étapes :
                \begin{itemize}
                    \item Pré-affrontement : \\
                        L'attaquant puis le défenseur appliquent les capacités de préaffrontement de leur choix.
                        Les capacités permanentes et de préaffrontement des factions tierces s'appliquent ensuite.
                    \item Affrontement : \\
                        Les deux opposants jettent simultanément leurs dés en fonction de leur score de combat (un dé par point).
                        Un jet de 6 humilie, un jet de 4 ou 5 blesse leur égo, le reste rate.
                        Les unités humiliées sont retirées du jeu et remis en réserve. C'est le joueur qui perd les unités qui décide quelles unités sont perdues.
                    \item Post-affrontement : \\
                        Les blessures et les capacités post-affrontement sont appliquées (de la même façon que les capacités pré-affrontement).
                        Une unité blessée doit se replier sur une zone adjacente où il n'y a pas d'unité de l'autre faction.
                        Si cela s'avère impossible, l'unité meurt.
                        L'attaquant effectue sa retraite en premier.
                \end{itemize}


        \subsection{Les productions majeure}

            Chaque production majeure (manga, light novel, …) a un effet unique.
            Lorsqu'une production majeure est créée, elle n'est sera jamais perdue, même si la condition de création n'est plus remplie.

        \subsection{Réserves}
            Chaque joueur dispose d'une réserve d'unités. En en dessinant une, elle est placée sur la carte et
            sort donc de la réserve. Il est impossible de dessiner une unité si on en pas en réserve. \newline
            Les licences disposent toutes d'un stock de 6 mascottes. Les stock des autres unités dépendent de
            la license.

        \subsection{Détermination de la victoire ou de la défaite}

            La partie peux se terminer dans deux cas :
            \begin{itemize}
                \item Un joueur possède plus de 30000 fans.
                \item L'évènement \textit{Crise économique du secteur} est déclenché.
            \end{itemize}

            Si le jeu s'est terminé pendant une phase de marketing, les joueurs n'ayant pas joués peuvent toujours organiser une dernière convention.
            Dans tout les cas, il est possible de révéler ses droits de campagne publicitaires.\newline
            Le vainqueur est celui qui a le plus de fans parmi ceux qui possèdent 6 productions.
            En cas d'égalité, les joueurs concernés se partagent la victoire. Si personne n'a les 6 productions,
            tout le monde perd, et les studios d'animation du monde entier font faillite.