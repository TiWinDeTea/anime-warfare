\chapter{Introduction}

    \section{Principe de base}

        Pour notre projet, nous avons décidé d'adapter le jeu de plateau Cthulhu Wars
        sous forme d'un jeu informatique, Anime Warfare. Le but de ce jeu est simple :
        être le premier à obtenir le maximum de fans tout en réalisant ses six productions
        majeures. Chaque joueur joue chacun son tour durant diverses phases, et peut obtenir
        des capacités uniques, liés à sa license (son ''équipe'').
        \newline
        Chaque joueur dispose d'unités qu'il peut, moyennant un coût, placer sur une
        carte ; les unités ainsi placée s'affrontent alors dans des batailles d'insultes
        sanglantes, dans le but d'humilier leurs adversaires à mort.

    \section{Principales différences avec Cthulhu Wars}

        L'univers étant différent de celui de base, les termes ont étés adaptés :
        \begin{itemize}
            \item Les Points de pouvoirs deviennent des employés du staff.
            \item Les Signes des anciens sont des droits de campagnes publicitaires.
            \item La Phase de l'apocalypse est représenté par la phase marketing.
            \item Les Rituels de l'apocalypses se transforment en convention.
            \item L'échelle de l'apocalypse est maintenant la piste marketing.
            \item Un portail d'invocation est désigné comme étant un studio.
            \item Les adorateurs sont connus comme étant des mascotes.
            \item Les unités en générales sont de simple personnages.
            \item Les grimoires sont des productions majeures.
            \item Les factions sont des Licenses.
        \end{itemize}

        Le style de jeu reste principalement le même, mais on notera toutefois une
        différence importante : l'apparition d'un deck global. \\
        Le premier joueur d'un tour devra parfois tirer une carte dans ce deck, en
        choisissant parmi ses trois premières cartes. Un effet va alors s'appliquer
        pendant un nombre donné de tours à tout les joueurs, influançant le déroulement
        de la partie.