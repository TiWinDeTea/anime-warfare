\chapter{Fonctionnalités}

Nous avons implémenté, entre autre, les fonctionnalités suivante :
\begin{itemize}
    \item Menu des options
    \begin{itemize}
        \item Nom du joueur
        \item Activation ou non des animations
        \item Langue (français, anglais)
        \item Sauvegarde de la configuration
        \item …
    \end{itemize}
    \item Activation automatique des capacités.
    \item Capacités
    \begin{itemize}
        \item Clémence
        \item Retraite forcée
        \item Mes fans, mes fans, mes fans
        \item …
    \end{itemize}
    \item Actions
    \begin{itemize}
        \item Capturer des unités
        \item Déplacement des unités
        \begin{itemize}
            \item Multiselection des unités
            \item Plusieurs déplacements possible en une fois
            \item Zones accessibles mises en surbrillance
        \end{itemize}
        \item Capturer des studios
        \item Abandonner des studios
        \item ...
    \end{itemize}
    \item Les buffs
    \begin{itemize}
        \item Gestions des points d'attaques
        \item Invincibilité
        \item ...
    \end{itemize}
    \item Deck global avec buffs aléatoires tirés
        en début de tour.
    \item Gestion des conditions d'arrêts du jeu.
    \item Popup d'informations diverses (général, résumé de bataille, …)
    \item Batailles
    \begin{itemize}
        \item Capacités pré-bataille
        \item Capacités post-bataille
        \item Sélection des morts
        \item Sélection des blessés.
    \end{itemize}
    \item Système de salon où les joueurs peuvent patienter et sélectionner leur faction.
    \item Création d'unités
    \begin{itemize}
        \item Conditions de création variées sur les héros
        \item Gestion dynamique des coûts des unités.
    \end{itemize}
    \item Système de « map » constituée de zones interconnectées.
    \item Jeu en réseau
    \begin{itemize}
        \item Thread séparé.
        \item Entre 2 et 4 joueurs
        \item Possibilité de se connecter à distance
        \item Découverte des serveurs
        \item Gestion des mots de passes
        \item Systeme de chat instantané
        \begin{itemize}
            \item Formattage du chat (gras, couleurs, ...)
            \item Intégré au salon et au jeu en lui-même
            \item Minimisable
        \end{itemize}
    \end{itemize}
    \item Système extensible de buffs et de capacités (extensible via des événements)
    \item Interface graphique (dans un thread séparé)
    \item Les éléments d'interface graphiques sont stockées au format \textbf{fxml} (dérivé du \emph{xml}) et chargés au lancement du
        programme (les menus, les popups, …).
    \item Fichiers de configuration \emph{.properties} pour définir les constantes et les chaînes de caractères
        dans les différentes langues.
\end{itemize}

