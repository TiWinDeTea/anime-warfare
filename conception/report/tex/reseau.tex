\chapter{Réseau} \label{chapter:reseau}

    \section{Diagramme de classe}

        \mfigure[H]{width=\linewidth}{cl_server.png}{Diagramme de classe de l'architecture serveur}{cl::server}

        La classe GameServer est un pont entre les clients et les joueurs (clients).
        Par le biais du réseau, elle reçoit les requêtes des clients qu'elle
        retransmet à la logique via des évenements. De la même manière,
        le serveur écoute les évenements de la logique pour les retransmettre
        aux clients du jeu. C'est aussi la responsabilité de cette classe de
        lancer le jeu lorsque tout les joueurs sont prêts.
        \newline
        La classe Room représente le salon de jeu, avec son nom et une liste
        des joueurs connectés, ainsi qu'un mot de passe éventuel. L'instance
        de cette classe présente dans GameServer est partagée avec l'instance
        de la classe UDPListener.
        \newline
        La classe UDPListener, enfin, sert à écouter les requêtes de joueurs
        non connectés. Ainsi, lorsqu'un client envoie les bonnes donnés à
        l'UDPListener via un \textit{broadcasting}, celui-ci répond au client
        en lui renvoyant la Room, en omettant le mot de passe. Le client peut
        alors tenter de se connecter au serveur.

    \section{Diagrammes de séquence}

        \mfigure[H]{width=\linewidth}{seq_joinGame}{Diagramme de séquence – rejoindre une partie}{seq::joinGame}
        Si le mot de passe est faux, il ne se passe rien et le client ne rejoint pas la partie

        \mfigure[H]{width=\linewidth}{seq_launchGame}{Diagramme de séquence – lancer un server}{seq::launchServer}
        S'il est impossible d'écouter sur un port donné (par ce qu'il est déjà utilisé, par exemple), le serveur
        se referme et notifie le "lanceur" de l'échec du lancement.

        \begin{landscape}
            \mfigure[H]{width=\linewidth}{seq_moveUnit}{Diagramme de séquence – déplacer une unité}{seq::moveUnit}
        \end{landscape}

